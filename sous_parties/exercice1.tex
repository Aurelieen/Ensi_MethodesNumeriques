\documentclass[a4paper,11pt]{article}

\usepackage[utf8]{inputenc}
\usepackage[T1]{fontenc}
\usepackage[francais]{babel}
\usepackage{amsmath,amssymb}
\usepackage{fullpage}
\usepackage{xspace}
\usepackage{graphicx}
\usepackage{verbatim}
\usepackage{listings}
\usepackage[usenames,dvipsnames]{color}
\usepackage{url}

\lstset{basicstyle=\small\tt,
  keywordstyle=\bfseries\color{Orchid},
  stringstyle=\it\color{Tan},
  commentstyle=\it\color{LimeGreen},
  showstringspaces=false}

\newtheorem{question}{Question}
\newtheorem{exo}{Exercice}

\newcommand{\dx}{\,dx}
\newcommand{\ito}{,\dotsc,}
\newcommand{\R}{\mathbb{R}}
\newcommand{\C}{\mathbb{C}}
\newcommand{\N}{\mathbb{N}}
\newcommand{\Poly}[1]{\mathcal{P}_{#1}}
\newcommand{\abs}[1]{\left\lvert#1\right\rvert}
\newcommand{\norm}[1]{\left\lVert#1\right\rVert}
\newcommand{\pars}[1]{\left(#1\right)}
\newcommand{\bigpars}[1]{\bigl(#1\bigr)}
\newcommand{\set}[1]{\left\{#1\right\}}

\title{Stage de Toussaint Ensimag 1A \\Compte-Rendu du Tp de méthodes numériques/Latex}
\author{Aurelien PEPIN \and Antonin KLOPP-TOSSER}
\date{Mai 2017}

% ===============
\begin{document}

\maketitle

\section {Exercice 1}

Debut de réponse à la question 1


$\frac{u_i^{(k+1)}-u_i^{(k)}}{\delta_t} = \frac{\theta}{\delta_{x}^{2}}(C_{i+\frac{1}{2}}u_{i+1}^{(k+1)} - (C_{i+\frac{1}{2}} + C_{i-\frac{1}{2}})u_{i}^{(k+1)}) +
C_{i-\frac{1}{2}}u_{i-1}^{(k+1)}) + \frac{1 - \theta}{\delta_{x}^{2}}(C_{i+\frac{1}{2}}u_{i+1}^{(k)} - (C_{i+\frac{1}{2}} + C_{i-\frac{1}{2}})u_{i}^{(k)}) +
C_{i-\frac{1}{2}}u_{i-1}^{(k)}))$

\vspace{1cm}

$u_i^{(k+1)}-u_i^{(k)} = \theta \mu(C_{i+\frac{1}{2}}u_{i+1}^{(k+1)} - (C_{i+\frac{1}{2}} + C_{i-\frac{1}{2}})u_{i}^{(k+1)}) +
C_{i-\frac{1}{2}}u_{i-1}^{(k+1)}) + (1 - \theta)\mu (C_{i+\frac{1}{2}}u_{i+1}^{(k)} - (C_{i+\frac{1}{2}} + C_{i-\frac{1}{2}})u_{i}^{(k)}) +
C_{i-\frac{1}{2}}u_{i-1}^{(k)}))$

\vspace{1cm}
Ce qui donne sous forme matricielle:
\vspace{1cm}
$(I + \theta \mu A) U^{(k+1)} = (I + (\theta - 1) \theta \mu A)U^{(k)} + \mu B^{(k)}$
\vspace{1cm}
avec
$A = \begin{pmatrix}
(C_{0+\frac{1}{2}} + C_{0-\frac{1}{2}}) & C_{0-\frac{1}{2}} & 0 & 0 \\
C_{1+\frac{1}{2}} & (C_{1+\frac{1}{2}} + C_{1-\frac{1}{2}}) & C_{1-\frac{1}{2}} & 0 \\
0 & ... & ... & 0\\
0 & 0 & C_{n+\frac{1}{2}} & (C_{n+\frac{1}{2}} + C_{n-\frac{1}{2}})
\end{pmatrix}$
\vspace{1cm}
$B{(k)}$ contient les termes $u_{0}$ que l'on a oubliés dans l'équation sur la première ligne de la matrice.

\vspace{1cm}
Déso c'est dégueu
\end{document}
