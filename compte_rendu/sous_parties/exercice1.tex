\documentclass[a4paper,11pt]{article}

\usepackage[utf8]{inputenc}
\usepackage[T1]{fontenc}
\usepackage[francais]{babel}
\usepackage{amsmath,amssymb}
\usepackage{fullpage}
\usepackage{xspace}
\usepackage{graphicx}
\usepackage{verbatim}
\usepackage{listings}
\usepackage[usenames,dvipsnames]{color}
\usepackage{url}

\lstset{basicstyle=\small\tt,
  keywordstyle=\bfseries\color{Orchid},
  stringstyle=\it\color{Tan},
  commentstyle=\it\color{LimeGreen},
  showstringspaces=false}

\newtheorem{question}{Question}
\newtheorem{exo}{Exercice}

\newcommand{\dx}{\,dx}
\newcommand{\ito}{,\dotsc,}
\newcommand{\R}{\mathbb{R}}
\newcommand{\C}{\mathbb{C}}
\newcommand{\N}{\mathbb{N}}
\newcommand{\Poly}[1]{\mathcal{P}_{#1}}
\newcommand{\abs}[1]{\left\lvert#1\right\rvert}
\newcommand{\norm}[1]{\left\lVert#1\right\rVert}
\newcommand{\pars}[1]{\left(#1\right)}
\newcommand{\bigpars}[1]{\bigl(#1\bigr)}
\newcommand{\set}[1]{\left\{#1\right\}}

\title{Stage de Toussaint Ensimag 1A \\Compte-Rendu du Tp de méthodes numériques/Latex}
\author{Aurelien PEPIN \and Antonin KLOPP-TOSSER}
\date{Mai 2017}

% ===============
\begin{document}

\maketitle

\section {Exercice 1}

Debut de réponse à la question 1


$\frac{u_i^{(k+1)}-u_i^{(k)}}{\delta_t} = \frac{\theta}{\delta_{x}^{2}}(C_{i+\frac{1}{2}}u_{i+1}^{(k+1)} - (C_{i+\frac{1}{2}} + C_{i-\frac{1}{2}})u_{i}^{(k+1)}) +
C_{i-\frac{1}{2}}u_{i-1}^{(k+1)}) + \frac{1 - \theta}{\delta_{x}^{2}}(C_{i+\frac{1}{2}}u_{i+1}^{(k)} - (C_{i+\frac{1}{2}} + C_{i-\frac{1}{2}})u_{i}^{(k)}) +
C_{i-\frac{1}{2}}u_{i-1}^{(k)}))$

\vspace{1cm}

$u_i^{(k+1)}-u_i^{(k)} = \theta \mu(C_{i+\frac{1}{2}}u_{i+1}^{(k+1)} - (C_{i+\frac{1}{2}} + C_{i-\frac{1}{2}})u_{i}^{(k+1)}) +
C_{i-\frac{1}{2}}u_{i-1}^{(k+1)}) + (1 - \theta)\mu (C_{i+\frac{1}{2}}u_{i+1}^{(k)} - (C_{i+\frac{1}{2}} + C_{i-\frac{1}{2}})u_{i}^{(k)}) +
C_{i-\frac{1}{2}}u_{i-1}^{(k)}))$

\vspace{1cm}
Ce qui donne sous forme matricielle:
\vspace{1cm}
$(I + \theta \mu A) U^{(k+1)} = (I + (\theta - 1) \theta \mu A)U^{(k)} + \mu B^{(k)}$
\vspace{1cm}
avec
$A = \begin{pmatrix}
(C_{0+\frac{1}{2}} + C_{0-\frac{1}{2}}) & C_{0-\frac{1}{2}} & 0 & 0 \\
C_{1+\frac{1}{2}} & (C_{1+\frac{1}{2}} + C_{1-\frac{1}{2}}) & C_{1-\frac{1}{2}} & 0 \\
0 & ... & ... & 0\\
0 & 0 & C_{n+\frac{1}{2}} & (C_{n+\frac{1}{2}} + C_{n-\frac{1}{2}})
\end{pmatrix}$
\vspace{1cm}
$B{(k)}$ contient les termes $u_{0}$ que l'on a oubliés dans l'équation sur la première ligne de la matrice.

\vspace{1cm}
Déso c'est dégueu


Question 13.
\newline
$J(x_d)$ est la distance entre $x_d$ et la cible.
On en déduit:
si $J(x_1) = min(J(x_1), J(x_2), J(x_3))$, alors $x_1$ est le plus près de de $x_{cible}$, doncon choisit l'intervalle $[a, x_2]$.
De même, si $J(x_2) = min(J(x_1), J(x_2), J(x_3))$, alors on choisit l'intervalle $[x_1, x_3]$.
et si $J(x_3) = min(J(x_1), J(x_2), J(x_3))$, alors on choisit l'intervalle $[x_2, b]$.

\section{RESTE}

    Soit $x = (x_1, x_2, ... , x_n)$
    \newline
    $x^TAx = \sum\limits_{i = 1}^{n} (C_{i+1/2} + C_{i-1/2}) x_i^2 - 2 \sum\limits_{i = 1}^{n-1} C_{i+1/2} x_i x_{i+1} $
    \newline
    $ = \sum\limits_{i = 1}^{n} C_{i+1/2} x_i^2 + \sum\limits_{j = 0}^{n-1} C_{j+1/2} x_{j+1}^2  + - 2 \sum\limits_{i = 1}^{n-1} C_{i+1/2} x_i x_{i+1} $ On pose j = i-1
    \newline
    $ = \sum\limits_{i = 1}^{n-1} C_{i+1/2} (x_i - x_{i+1})^2 + C_{1/2} x_{1/2}^2 + C_{n + 1/2} x_{n}^2 > 0 $
    \newline
    Comme $x^TAx > 0 \forall x \in \R ^n $ alors la matrice A est symétrique définie positive.
    \newline

    D'après (9):
    $ \frac{\partial}{\partial x}[C(x)\frac{\partial u}{\partial x}](x_i, t_k) = \frac{
    C_{i+1/2}u_{i+1}^{(k)} - (C_{i+1/2} + C_{i-1/2})u_i^{(k)}+C_{i-1/2}u_{i-1}^{k}}{\delta^{2}}$
    Pour résoudre (13), on résout 9 pour chaque coefficient $u_i^{k}$

    $$
    \overset{B} {
        \begin{pmatrix}
            u_0 C_{1/2} \\
            0 \\
            0 \\
            0 \\
        \end{pmatrix}
    }
    $$

    Question 6.

    D'après l'équation (9),
    \newline
    $ \frac{\partial }{\partial x} [C(x)\frac{\partial u }{\partial x}](x_i, t_k) \approx \frac{C_{i+1/2}u_{i+1}^{(k)} - (C_{i+1/2} + C_{i-1/2})u_i^{(k)} + C_{i-1/2}u_{i-1}^{(k)}}{\delta_x^2} $
    \newline
    Si on écrit ce système, $\forall i $ sous forme matricielle, on obtient:
    \newline
    $ \frac{\partial }{\partial x} [C(x)\frac{\partial u }{\partial x}] = \frac{1}{\delta ^2} (-A u +
    \overset{B} {
        \begin{pmatrix}
            u_0 C_{1/2} \\
            0 \\
            0 \\
            0 \\
        \end{pmatrix}
    })$
    car $u(-l) = u_0$
    avec $u = (u_1^{(k)}, ... , u_n^{(k)}) $
    et A la matrice trouvée à la question 1.
    \newline
    Donc A u = B, avec $B = (b_0, ..., b_n)$
    où, $b_0 = C_{1/2} u_0$
    et $b_i = 0$  $\forall i \in [2, n] $
    \newline

    Ce système admet une solution unique car A est définie et donc inversible.
    \newline

    Question 7.

    On résout l'équation:
    $$\left\{
    \begin{array}{ll}
      \frac{\partial}{\partial x}[C(x)\frac{\partial u}{\partial x}] = 0 \\
      u(-l) = u_0 \\
      u(l) = 0
      \end{array}
    \right.
    $$

    $  \frac{\partial}{\partial x}[C(x)\frac{\partial u}{\partial x}] = 0 $
    $  -\frac{1}{l} e^{-\frac{x}{l}}\frac{\partial u}{\partial x} + e^{-\frac{x}{l}} \frac{\partial^2 u }{\partial x^2} = 0 $
    $ \frac{\partial^2 u}{\partial x^2} - \frac{1}{l} \frac{\partial u}{\partial x} = 0$

    $u(x) = \alpha + \beta e^{\frac{x}{l}} $

    Calcul des constantes:
    $$\left\{
      \begin{array}{ll}
        u(-l) = \alpha + \beta e^{-1} = u_0 = 1 \\
        u(l) = \alpha + \beta e = 0
      \end{array}
    \right.
    $$

    $$\left\{
      \begin{array}{ll}
        \beta = \frac{1}{e^{-1} - e} \\
        \alpha = -\frac{e}{e^{-1} - e}
      \end{array}
    \right.
    $$

    Donc :
    $u(x) = -\frac{e}{e^{-1} - e} + \frac{1}{e^{-1} - e} e^{\frac{x}{l}} $

    Question 8.
    \newline

    $M = I+ \theta \mu A$ et $N = I+(\theta - 1)\mu A$
    \newline
    M et N sont donc des polynomes de matrices.
    \newline
    Donc, soit $x_A$ un vecteur propre de A et $\lambda_A$ une valeur propre de A,
    alors $x_A$ est aussi un vecteur propre de M avec comme valeur propre $M\lambda_A$
    \newline
    Les valeurs propres de M sont $\lambda_M = 1+\theta \mu \lambda_A$.
    \newline
    Les valeurs propres de N sont $\lambda_N = 1+(\theta - 1)\mu \lambda_A$
    \newline
    Les valeurs propres de $M^{-1}N$ sont $\frac{\lambda_N}{\lambda_M} = \frac{1+(\theta - 1)\mu \lambda_A}{1+\theta \mu \lambda_A}$ $\forall \lambda_A$.
    $ 1+(\theta - 1)\mu \lambda_A < 1+\theta \mu \lambda_A$
    \newline
    Donc $\rho(M^{-1}N) = max(\frac{\lambda_N}{\lambda_M}) < 1$
    \newline
    \newline
    \vspace{12ex}

    Question 9.

    $M U^{(k+1)} = NU^{(k)} + \mu B$
    Cette méthode de décomposition converge uniquement si $\rho(M^{-1}N) < 1$, ce qui est le cas ici(question 8).
    \newline
    Cette décomposition converge vers $Ax = b$, avec A = M - N.
    \newline
    $M - N = I + \theta \mu A - I - (\theta - 1) \mu A$
    $ = \mu A $
    \newline
    $\mu A x = \mu B$
    $A x = B$
    On retrouve ici les matrices A et B de la question (6).
\end{document}
